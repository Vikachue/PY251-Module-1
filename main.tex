% \documentclass{article}
% \begin{document}
% Physics is {\em really} fun, and $e^{i\pi} = -1$.
% 
% Physics is \bf really fun, and $e^{i\pi} = {-1}$.
% \end{document}

\documentclass[11pt]{article}
\begin{document}
% Begin discussion of Newton’s laws of motion
\section{Newton’s Laws}
Newton’s first law says: ‘‘Every body continues in its state
of rest, or of uniform motion in a right line, unless it is
compelled to change that state by forces impressed upon it.’’
Newton’s second law is
\begin{equation}
\vec F = m \vec a
\end{equation}
where $\vec F$ is the force on a particle of mass $m$, and
$\vec a$ is the particle’s acceleration.
% Begin discussion of Newton's Third Law.
\subsection{Newton's Third Law}
Newton's third law states: ''For every action, there is an equal and opposite reaction."
% Begin discussion of Maxwell’s equations
\section{Maxwell’s Equations}
Maxwell’s equations include Gauss’s law, which reads
\begin{equation}
\oint \vec E \cdot d\vec a = \frac{Q}{\epsilon_0}
\end{equation}
in integral form.

Ampere's Law:
\begin{equation}
    \vec \nabla \times \vec B = \mu_0 \vec J + \mu_0 \epsilon_0 \frac{\partial \vec E}{\partial t}
\end{equation}
in differential form.

In addition, here is Faraday's Law:
\begin{equation}
\oint \vec E \cdot d\vec s = -\frac{d\Phi}{dt}
\end{equation}
The final equation of Maxwell is $\vec \nabla \cdot \vec B = 0$.
\end{document}